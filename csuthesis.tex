

\documentclass{thesisthemecsu}

% 文章信息
\titleen{How to use the CTEX theme for thesis of PhD of Central South University}
\titlecn{(如何使用CTEX主题来排版中南大学的博士论文,二号黑体)}

\priormajor{一级学科名称(三号宋体)}
\minormajor{二级学科或三级学科名称(三号宋体)}
\interestmajor{材料动力学}
\author{your name(三号宋体)}
\supervisor{xxx教授}
\subsupervisor{xxx教授}
\department{粉末冶金研究院}
\thesisdate{year=2017,month=5}

% lipsum
\newcommand{\lipsum} {
    这只是一个随机文本~这只是一个随机文本~这只是一个随机文本~这只是一个随机文本~这只是一个随机文本~这只是一个随机文本~这只是一个随机文本~这只是一个随机文本~这只是一个随机文本~这只是一个随机文本~这只是一个随机文本~这只是一个随机文本~这只是一个随机文本~这只是一个随机文本~这只是一个随机文本~这只是一个随机文本~这只是一个随机文本~这只是一个随机文本~这只是一个随机文本~这只是一个随机文本~这只是一个随机文本~这只是一个随机文本~这只是一个随机文本~这只是一个随机文本~这只是一个随机文本~这只是一个随机文本~这只是一个随机文本~这只是一个随机文本~
}

\begin{document}

% 封面 %
% ------------------------------------%
\makecoverpage
\newpage
% !!! 页码设置不在模版中设置请手动设置
% 学位论文的页码编排为:正文和后置部分用阿拉伯数字编连续码,前置部分用罗马数字单独编连续码(封面除外)。
% 设置封面页后的页码格式 %
% ------------------------------------%
\pagenumbering{Roman} % 大写罗马字母
\setcounter{page}{1} % 从1开始编号页码

% 设置页眉和页脚 %
\pagestyle{fancy}
% jing: 正文以前部分无需页眉 %
\fancyhf{} % 清空原有格式
\renewcommand{\headrulewidth}{0pt}
% jing: 前置部分封面页无需页码,其他前置部分需要。
\fancyhf[CEF,COF]{\thepage} % 所有(奇数和偶数)中间页脚

% 扉页 %
% ------------------------------------%
\maketitlepage
\newpage
% 声明页
\announcement
\newpage
% 设置中文摘要
\keywordscn{每篇论文应选取3-8个关键词,每一关键词之间用分号分开,最后一个关键词后不打标点符号。}
\categorycn{中图分类号(http://www.ztflh.com/)和UDC号(《国际十进位分类法》)可在图书馆查阅获得。}
\begin{abstractcn}
\lipsum
\end{abstractcn}
\newpage
% 设置英文摘要
\keywordsen{Latex, CSU}
\categoryen{Classification}
\begin{abstracten}
Tex is a useful language for typesetting an article. In this article, the usage of this template is to be introduced.
\end{abstracten}
\newpage

% 正文 %
% --------------------------------------------%
% 设置页码格式 %
\setcounter{page}{1} % 重置目录页码为小写罗马字体
\pagenumbering{roman} % 设置页码为小写罗马字体
% 设置页眉和页脚 %
\pagestyle{fancy}
\renewcommand{\headrulewidth}{1pt}
\fancyhf[ROH,REH]{\small\leftmark} % 设置所有(奇数和偶数)右侧页眉
\fancyhf[LEH,LOH]{中南大学博士学位论文} % 设置所有(奇数和偶数)左侧页眉
\fancyhf[CEF,COF]{\thepage} % 所有(奇数和偶数)中间页脚

% 目录
% -------------------------------------------%
{
    \zihao{-4} \songti \tableofcontents
}
\newpage
% 正文内容 %
% --------------------------------------------%
\setcounter{page}{1} % 重置页码编号
\pagenumbering{arabic} % 设置页码编号为阿拉伯数字

% 可以使用include命令导入tex文件,从而避免过多修改本文件。%
% 可以使用include命令导入tex文件,从而避免过多修改本文件。%
% 可以使用include命令导入tex文件,从而避免过多修改本文件。%

% 论文正文是主体,主体部分应从另页右页开始,每一章应另起页。一般由序号标题、文字叙述、图、表格和公式等五个部分构成。
% 论文正文是主体,主体部分应从另页右页开始,每一章应另起页。一般由序号标题、文字叙述、图、表格和公式等五个部分构成。

\section{绪论} % 1.
\subsection{题名} % 1.1
\lipsum

\begin{figure}
    \begin{center}
        \includegraphics[width=7cm]{figures/csu.png}
        \bicaption{图}{中南大学校徽}{Fig}{Logo of Central South University}
        \label{fig:logo-csu}
    \end{center}
\end{figure}

\subsubsection{题名} % 1.1.1
\lipsum
\subsubsection{题名} % 1.1.2
\lipsum
\newpage

\section{题名}
\lipsum
\subsection{题名}
\subsubsection{题名}
\lipsum
\subsubsection{题名}
\lipsum
\subsection{题名}
\subsubsection{题名}
\lipsum
\subsubsection{题名}
\lipsum
\subsection{题名}
\subsubsection{题名}
\lipsum
\subsubsection{题名}
\lipsum
\newpage

\section{题名}
\subsection{题名}
\subsubsection{题名}
\lipsum
\subsubsection{题名}
\lipsum
\subsection{题名}
\subsubsection{题名}
\lipsum
\subsubsection{题名}
\lipsum
\subsection{题名}
\subsubsection{题名}
\lipsum
\subsubsection{题名}
\lipsum
\newpage

\section{结论}
\lipsum
\newpage

% 以下为设置无编号章节标题 %
% -------------------------------------------------------%
% https://www.zhihu.com/question/29413517/answer/44358389 %
% 说明如下:
% secnumdepth 这个计数器是 LaTeX 标准文档类用来控制章节编号深度的。在 article 中,这个计数器的值默认是 3,对应的章节命令是 \subsubsection。也就是说,默认情况下,article 将会对 \subsubsection 及其之上的所有章节标题进行编号,也就是 \part, \section, \subsection, \subsubsection。LaTeX 标准文档类中,最大的标题是 \part。它在 book 和 report 类中的层级是「-1」,在 article 类中的层级是「0」。这里,我们在调用 \appendix 的时候将计数器设置为 -2,因此所有的章节命令都不会编号了。不过,一般还是会保留 \part 的编号的。所以在实际使用中,将它设置为 0 就可以了。

% 作者:知乎用户
% 链接:https://www.zhihu.com/question/29413517/answer/44391211
% 来源:知乎
% 著作权归作者所有。商业转载请联系作者获得授权,非商业转载请注明出处。
% 在修改过程中请注意不要破环命令的完整性

\renewcommand\appendix{\setcounter{secnumdepth}{-2}}
\appendix
% 参考文献列表
\section{参考文献}
\nocite{*} % 此处用于列出所有参考文献
\bibliography{ref} % 参考文献源
\newpage

\section{攻读学位期间主要研究成果} % 无章节编号
\lipsum
\newpage

\section{致谢} % 无章节编号
\lipsum
\newpage

\end{document}
