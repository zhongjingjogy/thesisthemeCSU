% 论文正文是主体,主体部分应从另页右页开始,每一章应另起页。一般由序号标题、文字叙述、图、表格和公式等五个部分构成。

\section{绪论} % 1.
\subsection{题名} % 1.1
\lipsum

\begin{figure}
    \begin{center}
        \includegraphics[width=7cm]{figures/csu.png}
        \bicaption{图}{中南大学校徽}{Fig}{Logo of Central South University}
        \label{fig:logo-csu}
    \end{center}
\end{figure}

\subsubsection{题名} % 1.1.1
\lipsum
\subsubsection{题名} % 1.1.2
\lipsum
\newpage

\section{题名}
\lipsum
\subsection{题名}
\subsubsection{题名}
\lipsum
\subsubsection{题名}
\lipsum
\subsection{题名}
\subsubsection{题名}
\lipsum
\subsubsection{题名}
\lipsum
\subsection{题名}
\subsubsection{题名}
\lipsum
\subsubsection{题名}
\lipsum
\newpage

\section{题名}
\subsection{题名}
\subsubsection{题名}
\lipsum
\subsubsection{题名}
\lipsum
\subsection{题名}
\subsubsection{题名}
\lipsum
\subsubsection{题名}
\lipsum
\subsection{题名}
\subsubsection{题名}
\lipsum
\subsubsection{题名}
\lipsum
\newpage

\section{结论}
\lipsum
\newpage

% 以下为设置无编号章节标题 %
% -------------------------------------------------------%
% https://www.zhihu.com/question/29413517/answer/44358389 %
% 说明如下:
% secnumdepth 这个计数器是 LaTeX 标准文档类用来控制章节编号深度的。在 article 中,这个计数器的值默认是 3,对应的章节命令是 \subsubsection。也就是说,默认情况下,article 将会对 \subsubsection 及其之上的所有章节标题进行编号,也就是 \part, \section, \subsection, \subsubsection。LaTeX 标准文档类中,最大的标题是 \part。它在 book 和 report 类中的层级是「-1」,在 article 类中的层级是「0」。这里,我们在调用 \appendix 的时候将计数器设置为 -2,因此所有的章节命令都不会编号了。不过,一般还是会保留 \part 的编号的。所以在实际使用中,将它设置为 0 就可以了。

% 作者:知乎用户
% 链接:https://www.zhihu.com/question/29413517/answer/44391211
% 来源:知乎
% 著作权归作者所有。商业转载请联系作者获得授权,非商业转载请注明出处。
% 在修改过程中请注意不要破环命令的完整性

\renewcommand\appendix{\setcounter{secnumdepth}{-2}}
\appendix
% 参考文献列表
\section{参考文献}
\nocite{*} % 此处用于列出所有参考文献
\bibliography{ref} % 参考文献源
\newpage

\section{攻读学位期间主要研究成果} % 无章节编号
\lipsum
\newpage

\section{致谢} % 无章节编号
\lipsum
\newpage